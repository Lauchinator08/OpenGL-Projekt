\documentclass{article}
\begin{document}
	\section*{Dokumentation OpenGL-Projekt Maximilian Maag}
	\subsection*{Aufgabe 1}
	Vollständig Bearbeitet mit FPS anzeige.
	\subsection*{Aufgabe 2}
	\subsection*{a)}
	Würfel eingefügt mit Code in der Methode displayScene.
	\subsection*{b)}
	Zweiter Würfel dreht doppelt so schnell und ist grün.
	\subsection*{c)}
	Zeichnen der Pyramide durch eine Methode triangle, welche zunächst Dreiecke zeichnet. Anschließend wird die Pyramide aus Aufrufen von quad und triangle zusammengesetzt und in der Funktion displayScene eingefügt.
	\subsection*{d)}
	Pyramide in Rot eingefügt.
	\subsection*{e)}
	Pyramide in blau eingefügt.
	\subsection*{f)}
	Alle drei Pyramiden können sich drehen. 
	\subsection*{Aufgabe 3}
	\subsection*{a) + b) + c)}
	Die Kammerpositionen werden über einen Switchcase abgefangen.
	\\ \\
	Die folgenden Antworten sind Mutmaßungen auf Basis meines Verständnisses für Optik und Physik, da ich das Bild im Browser sehr schlecht erkennen kann.
	\subsection*{d)}
	Es ergibt sich eine Sicht ähnlich wie durch ein Weitwinkelobjektiv. Stichwort Fischauge.
	\subsection*{e)}
	\subsection*{f)}
	Bild wird verzerrt, da es in 16:9 gerendert wird aber in einem quadratischen Panel dargestellt wird.
\end{document}